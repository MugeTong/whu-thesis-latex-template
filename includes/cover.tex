%% cover.tex - 封面单独文件
% 设置本页的页眉页脚为空, 避免页码出现
\thispagestyle{empty}
% 封面页加到 PDF 书签
\pdfbookmark[0]{封面}{title}
% 封面页重新定义页边距
\newgeometry{
    top=2.5truecm,
    bottom=2.7truecm,
    left=3truecm,
    right=3truecm,
}
\begin{center}  % 居中设置
\vspace*{2.1cm}  % 强制插入顶部空白

% 武汉大学校徽
\ifanonymous
    \includegraphics[width=0.33\textwidth]{whu-logo-anonymous.png}
\else
    \includegraphics[width=0.33\textwidth]{whu-logo.png}
\fi
\vspace{1.34cm}

%  本科毕业论文(设计)文本
{\songti \zihao{2} \textbf{本科毕业论文(设计)}}
\vspace{3.25cm}

% 论文标题
{\kaishu \zihao{1} \the\TitleChinese}
\vfill  % 填充中间的空白

% 记录最长的字符串长度
\newlength{\covermaxlen}
\newlength{\covertestlen}
% 处理匿名信息
\ifanonymous
    \AuthorChinese={}
    \MajorChinese={}
    \StudentNumber={}
    \DepartmentChinese={}
    \SupervisorChinese={}
\fi

% 作者信息
{\songti \zihao{4}
    % 计算最长字符串长度
    \settowidth{\covermaxlen}{\the\AuthorChinese}
    \settowidth{\covertestlen}{\the\MajorChinese}
    \ifthenelse{\covertestlen>\covermaxlen}{\setlength{\covermaxlen}{\covertestlen}}{}
    \settowidth{\covertestlen}{\the\StudentNumber}
    \ifthenelse{\covertestlen>\covermaxlen}{\setlength{\covermaxlen}{\covertestlen}}{}
    \settowidth{\covertestlen}{\the\DepartmentChinese}
    \ifthenelse{\covertestlen>\covermaxlen}{\setlength{\covermaxlen}{\covertestlen}}{}
    \settowidth{\covertestlen}{\the\SupervisorChinese}
    \ifthenelse{\covertestlen>\covermaxlen}{\setlength{\covermaxlen}{\covertestlen}}{}

    % 创建表格
    \begin{tabular}{c p{5.2cm}<{\centering}}
        \makebox[5\ccwd][s]{姓 \hfill 名:} & \makebox[\covermaxlen][s]{\the\AuthorChinese} \\
        \makebox[5\ccwd][s]{学 \hfill 号:} & \makebox[\covermaxlen]{\the\StudentNumber} \\
        \makebox[5\ccwd][s]{专 \hfill 业:} & \makebox[\covermaxlen][s]{\the\MajorChinese} \\
        \makebox[5\ccwd][s]{学 \hfill 院:} & \makebox[\covermaxlen][s]{\the\DepartmentChinese} \\
        \makebox[5\ccwd][s]{指 \hfill 导 \hfill 老 \hfill 师:} & \makebox[\covermaxlen][s]{\the\SupervisorChinese} \\[1ex]
    \end{tabular}
}
\vspace{4cm}

% 日期
{\songti \zihao{4} \ziju{0.2} \the\DateChinese}

\end{center}

% 恢复cls文件中定义的页边距
\restoregeometry
